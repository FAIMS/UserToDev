\startcomponent[DeployModules]
\starttext
\section[how-to-deploy-modules]{How to Deploy Modules}

\subsection[opening-the-faims-server-web-interface]{Opening the FAIMS Server Web Interface}

Now that your FAIMS server is set up, you should be able to access it from any device on the same network as the server. On a computer, navigate to the IP address that appears in the browser on your FAIMS Ubuntu installation.

\placefigure[force][server-IP]{Enter "http://" and the IP address to access your faims server.}{\externalfigure[image28.png][width=.5\textwidth]}

Next, log into the FAIMS server using either the default admin account or an account that you set up (we recommend setting up individual accounts for each user). {\bf Reminder:} The default FAIMS account is "faimsadmin@intersect.org.au", password "Pass.123".

\placefigure[force][server-login]{This sign-in page appears upon accessing your FAIMS server.}{\externalfigure[image29.png][width=.5\textwidth]}

\subsection[uploading-build-files-to-the-faims-server]{Uploading Build Files to the FAIMS Server}

As we've already explained, this part is very simple. All you need to do is:

\startitemize[n][stopper=.]
\item Login to the Server
\item Click 'Show Modules'
\item Click 'Create Module'
\item Give your Module a name, and enter as much metadata as possible.
\item Upload the Data Definition Schema (data_schema.xml).
\item Upload the User Interface Schema (ui_schema.xml)
\item Upload the User Interface Logic (ui_logic.bsh)
\item Upload the Translation (Arch16n) file (faims.properties)
\stopitemize

Let it combine and process them into a module. Now you've created a module which will live on the server. 

On your Android device(s), open the FAIMS app and connect to your server. When you see your module, download and open it. 

Voila!

\quiz{Test your knowledge}{
  \item Where can you find the IP address of your FAIMS server?
  \item Do you download your module wirelessly from the server, or directly to your device from a computer?
}


\stoptext