\startcomponent[PartsOfModule]
\starttext


In the next chapter, you're going to make your module by following these steps:

\startitemize[packed]
\item Learn what the components of a module are and what they do.
\item Download tools that allow you to create module components (or \quotation{necessary files}) based on a set of instructions.
\item Learn how to write those instructions so that the necessary files you produce will fit together to assemble the module you need.
\item Learn how to set up and operate the tools so that they'll follow your instructions and make the necessary files you need.
\item Use software to hunt down and correct any mistakes you've made.
\item Send the necessary files to the FAIMS servers and create a module you and your team can download and use.
\item If you need to modify parts of your module, create new necessary files and send them to the FAIMS server. They'll replace the old ones and allow everyone on your team to update to the new, improved version.
\stopitemize

\section{The Parts of a Module}

Modules are assembled by your FAIMS server from components called \quotation{necessary files}\footnote{Brian's note: In our academic papers about modules we will sometimes use the term \quotation{Definition packets.} The module is the product of the processing of the \quotation{necessary files} which as a set comprise the \quotation{definition packet}.}.

Speaking practically, most \quotation{necessary files} are text files. Unless noted, they tend to end with the file extension {\tt .xml} and can be opened and even edited with simple text editors, such as Notepad (although you may want to find something with a few more features, as we'll explain in the next section). Each necessary file serves a definite and distinct function in the final operation of the module.

Your team members can use a module without ever learning exactly what necessary files are or what they do, but you'll need to become familiar with them if you plan to make a module or alter one already in use. Here are the kinds of necessary files you'll come across working with FAIMS.

\subsection{Data Schema}

This file, which should appear on your computer as {\tt data_schema.xml}, defines what kinds of data you want to record and how they're related to each other. We go into a little more technical detail about what a Data Schema is and does in the section below, \about[tour-from-the-data-schema].

The Data Schema is one of the most fundamental and important necessary files of a FAIMS module. Unlike other necessary files, which can be replaced and updated even after the module's in use by your team, the Data Schema cannot be replaced. If for some reason your Data Schema no longer provides satisfactory results, you'll need to create a whole new module and instruct your team to transition over to it. This is the one part you should be absolutely certain you're happy with before you proceed.

\subsection{UI Schema}

The User Interface (UI) Schema, or {\tt ui_schema.xml}, defines what your module will actually look like and where your users will input their data.

\subsection[validation-schema]{Validation Schema}

The Validation Schema, {\tt validation.xml}, defines what kinds of data your team {\em should} be collecting. It allows the module to \quotation{validate,} or give a thumbs-up or thumbs-down to, your team's submissions to make sure the data being collected is thorough enough or makes sense.

For example, let's say your team is submitting data on handaxes they've excavated from a particular context. To complete your research goals you need to make sure that every time someone records a handaxe, they report how much it weighs in grams.

When you're designing your module, you will utlimately create a Validation Schema that ensures users must:

\startitemize[a, packed, joinedup]
\item put something in \quotation{Weight of Handaxe} instead of leaving it blank
\item enter a number, not a phrase
\item list the weight in grams, not pounds, tons, or catties.
\stopitemize

If a module checks a submission and discovers it isn't valid, it alerts the user who made the error, flagging the incomplete or problematic field as \quotation{dirty} and giving the user some idea what the problem is. However, the data are still collected and can be viewed or modified by the project manager.

\subsection[ui-logic]{UI Logic}

The UI Logic performs a few functions. It tells a module's user interface how to behave, governs operations on the database, and facilitates interactions between FAIMS and devices such as GPS receivers, cameras, and bluetooth-compatible peripherals.

For example, when a record is created, the user who created it and a record of when it was created (also called a \quotation{timestamp}) are automatically stored in the database. A UI Logic program can be used to 1) query the database to retrieve either of these points of data, and; 2) update the UI to display the retrieved data to the user.

\subsection[arch16n]{Arch16n}

You won't necessarily need to mess with your module's Arch16n file. It's there to allow you to provide synonyms and translations for the \quotation{entities} in your module--useful if some of your team members speak a different language or use different terminology, in which case they would have their version of the module translated automatically.

\subsection[css]{CSS}

The UI Schema defines the basic layout of your module's user interface, but the details, like how the entry fields and controls appear, are defined by the Cascading Style Sheet (CSS). You set these styles using the {\tt ui_styling.css} file.

\subsection[picture-gallery-images]{Picture Gallery Images}

This part you handle more directly. Simply sort your images into folders, then put them all in a tarball (see \about[sharedata] for instructions). You can upload the tarball to the module generator directly, same as any other necessary file.
\quiz{Test your knowledge of some FAIMS jargon.}{

\item If data entered by a user isn't valid, is it collected?

\item What {\em necessary file} cannot be altered once a module has been created?

\item Which {\em necessary file} do you need to worry about if some of your team speaks English and some only French?
}



\stoptext
\stopcomponent