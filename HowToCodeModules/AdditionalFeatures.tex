\startcomponent[AdditionalFeatures]
\starttext

\section[additional-features]{Additional Features}

In no particular order, here are some other useful things you can do for your module in module.xml.

\subsection[add-a-picture-gallery]{Add a Picture Gallery}

Remember when we told you how to add a tarball of images to your module? If you use the element type "picture," you can allow user selectable options to be displayed as these images. Take a look at this tab here:

\placecode[here][codePictureGallery]{XML for a picture gallery.}{
\startframedtext[align={hanging, hz},
				 width=\makeupwidth,
				background=color,
				backgroundcolor=gray,				
				framecolor=gray,
				rulethickness=1pt				
				 ]
\starttyping[option=xml, tab=4, style={\tfx\tt}] 
	<Script t="picture">
		<desc>Type of used script.</desc>
		<opts>
			<opt p="picture1.jpg">Archaic-Epichoric</opt>
			<opt p="picture2.jpg">Old-Attic</opt>
			<opt p="picture3.jpg">Ionic</opt>
			<opt p="picture4.jpg">Roman</opt>
			<opt p="picture5.jpg">Indistinguishable</opt>
			<opt p="picture6.jpg">Other</opt>
		</opts>
	</Script>
\stoptyping
\stopframedtext
}

Each picture is referenced by the tag, then labeled with the non-tagged text inside the element.

{\em Hierarchical Dropdown}

"Hierarchical dropdown" is a fancy way of saying "selecting one option sometimes brings up more specific options." You do this very simply, by including <opt> tags inside of other <opt> tags. For example:

\placecode[here][codeHierarchicalDropdown]{XML for hierarchical dropdowns.}{
\startframedtext[align={hanging, hz},
				 width=\makeupwidth,
				background=color,
				backgroundcolor=gray,				
				framecolor=gray,
				rulethickness=1pt				
				 ]
\starttyping[option=xml, tab=4, style={\tfx\tt}] 
	<Script t="dropdown">
		<desc>Type of used script.</desc>
		<opts>
			<opt>Archaic-Epichoric
				<opt>A specific type of archaic-epichoric script
					<opt>An even more specific type of that specific type archaic-epichoric script</opt>
				</opt>
				<opt>Another type of archaic-epichoric script</opt>
			</opt>
			<opt>Old-Attic</opt>
			<opt>Ionic</opt>
			<opt>Roman</opt>
			<opt>Indistinguishable</opt>
			<opt>Other</opt>
		</opts>
	</Script>
\stoptyping
\stopframedtext
}

\subsection[using-the-translation-file]{Using the Translation File}

\subsection[autonumbering]{Autonumbering}

Basic autonumbering can be achieved using a combination of the f="autonum" flag and the <autonum/> tag. By flagging an input with "autonum", one indicates to the FAIMS-Tools that the ID of the next created entity---the entity containing the flagged field---should be taken from the corresponding field generated using the <autonum/> tag. For instance the Creatively_Named_ID in the below module will take its values from a field in Control which is generated by the use of the <autonum/> tag.

\placecode[here][codeAutoNumbering]{XML for autonumbering.}{
\startframedtext[align={hanging, hz},
				 width=\makeupwidth,
				background=color,
				backgroundcolor=gray,				
				framecolor=gray,
				rulethickness=1pt				
				 ]
\starttyping[option=xml, tab=4, style={\tfx\tt}] 
	<module>
		<Control>
			<Control>
				<Create_Entity t="button" l="Tab_Group" />
				<autonum/>
			</Control>
		</Control>
		<Tab_Group>
			<Tab>
				<Creatively_Named_ID f="id autonum" />
			</Tab>
		</Tab_Group>
	</module>
\stoptyping
\stopframedtext
}

The field will appear to the user as "Next Creatively Named ID" and will initially be populated with the number 1. When the user creates a Tab_Group entity, it will take that number as its "Creatively Named ID". The "Next Creatively Named ID" will then be incremented to 2, ready to be copied when a subsequent Tab_Group entity is created.

Multiple fields can be flagged as being autonumbered like so:

\placecode[here][codeMultipleAutoNum]{XML for autonumbering multiple fields.}{
\startframedtext[align={hanging, hz},
				 width=\makeupwidth,
				background=color,
				backgroundcolor=gray,				
				framecolor=gray,
				rulethickness=1pt				
				 ]
\starttyping[option=xml, tab=4, style={\tfx\tt}] 
	<module>
		<Control>
			<Control>
				<Create_Entity t="button" l="Tab_Group" />
				<autonum/>
			</Control>
		</Control>
		<Tab_Group>
			<Tab>
				<Creatively_Named_ID f="id autonum" />
				<Creatively_Named_ID_2 f="id autonum" />
			</Tab>
		</Tab_Group>
		<Other_Tab_Group>
			<Tab>
				<Creatively_Named_ID_3 f="id autonum" />
			</Tab>
		</Other_Tab_Group>
	</module>
\stoptyping
\stopframedtext
}

\subsection[restricting-data-entry-to-decimals-for-a-field]{Restricting Data Entry to Decimals for a Field}

-Single flag to denote as a number field

\subsection[type-guessing-for-gui-elements-in-faims-tools]{Type Guessing for GUI Elements in FAIMS-Tools}

The FAIMS-Tools generate.sh program will attempt to make a reasonable assumption about what the t attribute should be set to if it is omitted from a GUI element's set of XML tags.

If the XML tags do not contain a set of <opts> tags nor the f="user" flag, t="input" is assumed. Example:

\placecode[here][codeInputAssumed]{Without <opts> or f="user", t="input" is assumed.}{
\startframedtext[align={hanging, hz},
				 width=\makeupwidth,
				background=color,
				backgroundcolor=gray,				
				framecolor=gray,
				rulethickness=1pt				
				 ]
\starttyping[option=xml, tab=4, style={\tfx\tt}] 
	<Entity_Identifier f="id"/> <!-- This'll be an input -->
\stoptyping
\stopframedtext
}

If the XML tag is flagged with f="user", t="dropdown" is assumed. Example:
\placecode[here][codeDropdownAssumed]{When flagged with f="user", t="dropdown" is assumed.}{
\startframedtext[align={hanging, hz},
				 width=\makeupwidth,
				background=color,
				backgroundcolor=gray,				
				framecolor=gray,
				rulethickness=1pt				
				 ]
\starttyping[option=xml, tab=4, style={\tfx\tt}] 
	<List_of_Users f="user"/> <!-- This'll be a dropdown -->
\stoptyping
\stopframedtext
}

If the XML tags contain an <opts> element and no descendants with p attributes, t="list" is assumed. Example:

\placecode[here][codeListAssumed]{Tags containing <opts> with no descendants with p attributes are assumed to be lists.}{
\startframedtext[align={hanging, hz},
				 width=\makeupwidth,
				background=color,
				backgroundcolor=gray,				
				framecolor=gray,
				rulethickness=1pt				
				 ]
\starttyping[option=xml, tab=4, style={\tfx\tt}] 
	<Element> <!-- This'll be a list -->
		<opts>
			<opt>Option 1</opt>
			<opt>Option 2</opt>
		</opts>
	</Element>
\stoptyping
\stopframedtext
}

If the XML tags contain an <opts> element and one or more descendants with p attributes, t="picture" is assumed. Example:

\placecode[here][codePictureAssumed]{Tags containing <opts> and one or more descendants with p attributes are assumed to be picture galleries.}{
\startframedtext[align={hanging, hz},
				 width=\makeupwidth,
				background=color,
				backgroundcolor=gray,				
				framecolor=gray,
				rulethickness=1pt				
				 ]
\starttyping[option=xml, tab=4, style={\tfx\tt}] 
	<Element> <!-- This'll be a picture gallery -->	
		<opts>
			<opt p="Lovely_Image.jpg>Option 1</opt>
			<opt >Option 2</opt>
		</opts>
	</Element>
\stoptyping
\stopframedtext
}

Finally, if the XML tags have the "ec" attribute, t="list" is assumed.

There are arguments both for and against the use of the type guessing feature because, while improving succinctness, it also makes the module.xml file less intelligible to uninitiated programmers. Because of this, the FAIMS-Tools will display warnings when a module is generated from an XML file whose GUI elements are missing their t attributes. These can be hidden by adding suppressWarnings="true" to the opening <module> tag like so:

\placecode[here][codeSuppressWarnings]{XML to suppress warnings.}{
\startframedtext[align={hanging, hz},
				 width=\makeupwidth,
				background=color,
				backgroundcolor=gray,				
				framecolor=gray,
				rulethickness=1pt				
				 ]
\starttyping[option=xml, tab=4, style={\tfx\tt}] 
	<module suppressWarnings="true">
		<!-- Tab groups go here... -->
	</module>
\stoptyping
\stopframedtext
}

\subsection[annotation-and-certainty]{Annotation and Certainty}

\subsection[exporting-data]{Exporting Data}

\stoptext
\stopcomponent