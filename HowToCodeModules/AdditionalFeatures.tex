\startcomponent[AdditionalFeatures]
\starttext

\section[additional-features]{Additional Features}

In no particular order, here are some other useful things you can do for your module in {\tt module.xml}.

\subsection[add-a-picture-gallery]{Add a Picture Gallery}

Remember when we told you how to add a tarball of images to your module? If you use the element type \quotation{picture,} you can allow user selectable options to be displayed as these images. Take a look at the tab in \in{Codeblock}[codePictureGallery]: Each picture is referenced by the tag, then labeled with the non-tagged text inside the element.

\placecode[here][codePictureGallery]{XML for a picture gallery.}{
\startXML
<Script t="picture">
	<desc>Type of used script.</desc>
	<opts>
		<opt p="picture1.jpg">Archaic-Epichoric</opt>
		<opt p="picture2.jpg">Old-Attic</opt>
		<opt p="picture3.jpg">Ionic</opt>
		<opt p="picture4.jpg">Roman</opt>
		<opt p="picture5.jpg">Indistinguishable</opt>
		<opt p="picture6.jpg">Other</opt>
	</opts>
</Script>		
\stopXML
}


\subsection[hierarchical-dropdown]{Hierarchical Dropdown}

 \quotation{Hierarchical dropdown} is a fancy way of saying \quotation{selecting one option sometimes brings up more specific options.} You do this very simply, by including {\tt <opt>} tags inside of other {\tt <opt>} tags. For example, \in{Codeblock}[codeHierarchicalDropdown].

\placecode[here][codeHierarchicalDropdown]{XML for hierarchical dropdowns.}{
\startXML
<Script t="dropdown">
	<desc>Type of used script.</desc>
	<opts>
		<opt>Archaic-Epichoric
			<opt>A specific type of archaic-epichoric script
				<opt>An even more specific type of that specific type archaic-epichoric script</opt>
			</opt>
			<opt>Another type of archaic-epichoric script</opt>
		</opt>
		<opt>Old-Attic</opt>
		<opt>Ionic</opt>
		<opt>Roman</opt>
		<opt>Indistinguishable</opt>
		<opt>Other</opt>
	</opts>
</Script>		
\stopXML
}

\subsection[using-the-translation-file]{Using the Translation File}

\subsection[autonumbering]{Autonumbering}

Basic autonumbering can be achieved using a combination of the {\tt f="autonum"} flag and the {\tt <autonum/>} tag. By flagging an input with {\tt autonum}, one indicates to the FAIMS-Tools that the ID of the next created entity---the entity containing the flagged field---should be taken from the corresponding field generated using the {\tt <autonum/>} tag. For instance the Creatively_Named_ID in \in{Codeblock}[codeAutoNumbering] will take its values from a field in Control which is generated by the use of the {\tt <autonum/>} tag.

\placecode[here][codeAutoNumbering]{XML for autonumbering.}{
\startXML
<module>
	<Control>
		<Control>
			<Create_Entity t="button" l="Tab_Group" />
			<autonum/>
		</Control>
	</Control>
	<Tab_Group>
		<Tab>
			<Creatively_Named_ID f="id autonum" />
		</Tab>
	</Tab_Group>
</module>		
\stopXML
}

The field will appear to the user as {\tt Next Creatively Named ID} and will initially be populated with the number 1. When the user creates a Tab_Group entity, it will take that number as its {\tt Creatively Named ID}. The {\tt Next Creatively Named ID} will then be incremented to 2, ready to be copied when a subsequent {\tt Tab_Group} entity is created.

Multiple fields can be flagged as being autonumbered as in \in{Codeblock}[codeMultipleAutoNum].

\placecode[here][codeMultipleAutoNum]{XML for autonumbering multiple fields.}{
\startXML
<module>
	<Control>
		<Control>
			<Create_Entity t="button" l="Tab_Group" />
			<autonum/>
		</Control>
	</Control>
	<Tab_Group>
		<Tab>
			<Creatively_Named_ID f="id autonum" />
			<Creatively_Named_ID_2 f="id autonum" />
		</Tab>
	</Tab_Group>
	<Other_Tab_Group>
		<Tab>
			<Creatively_Named_ID_3 f="id autonum" />
		</Tab>
	</Other_Tab_Group>
</module>		
\stopXML
}

\subsection[restricting-data-entry-to-decimals-for-a-field]{Restricting Data Entry to Decimals for a Field}

-Single flag to denote as a number field

\subsection[type-guessing-for-gui-elements-in-faims-tools]{Type Guessing for GUI Elements in FAIMS-Tools}

The FAIMS-Tools generate.sh program will attempt to make a reasonable assumption about what the t attribute should be set to if it is omitted from a GUI element's set of XML tags.

If the XML tags do not contain a set of {\tt <opts>} tags nor the {\tt f="user"} flag, {\tt t="input"} is assumed. For example, \in{Codeblock}[codeInputAssumed].

\placecode[here][codeInputAssumed]{Without {\tt <opts>} or f="user", t="input" is assumed.}{
\startXML
<Entity_Identifier f="id"/> <!-- This'll be an input -->
\stopXML
}

If the XML tag is flagged with {\tt f="user"}, {\tt t="dropdown"} is assumed as in \in{Codeblock}[codeDropdownAssumed].

\placecode[here][codeDropdownAssumed]{When flagged with {\tt f="user"}, {\tt t="dropdown"} is assumed.}{
\startXML
<List_of_Users f="user"/> <!-- This'll be a dropdown -->
\stopXML
}

If the XML tags contain an {\tt <opts>} element and no descendants with p attributes, {\tt t="list"} is assumed. For example, \in{Codeblock}[codeListAssumed].

\placecode[here][codeListAssumed]{Tags containing {\tt <opts>} with no descendants with p attributes are assumed to be lists.}{
\startXML
<Element> <!-- This'll be a list -->
	<opts>
		<opt>Option 1</opt>
		<opt>Option 2</opt>
	</opts>
</Element>
\stopXML
}

If the XML tags contain an {\tt <opts>} element and one or more descendants with p attributes, {\tt t="picture"} is assumed. For example, \in{Codeblock}[codePictureAssumed].

\placecode[here][codePictureAssumed]{Tags containing {\tt <opts>} and one or more descendants with p attributes are assumed to be picture galleries.}{
\startXML
	<Element> <!-- This'll be a picture gallery -->	
		<opts>
			<opt p="Lovely_Image.jpg>Option 1</opt>
			<opt >Option 2</opt>
		</opts>
	</Element>
\stopXML
}

Finally, if the XML tags have the {\tt ec} attribute, {\tt t="list"} is assumed.

There are arguments both for and against the use of the type guessing feature because, while improving succinctness, it also makes the {\tt module.xml} file less intelligible to uninitiated programmers. Because of this, the FAIMS-Tools will display warnings when a module is generated from an XML file whose GUI elements are missing their t attributes. These can be hidden by adding suppressWarnings="true" to the opening {\tt <module>} tag as in \in{Codeblock}[codeSuppressWarnings].

\placecode[here][codeSuppressWarnings]{XML to suppress warnings.}{
\startXML
	<module suppressWarnings="true">
		<!-- Tab groups go here... -->
	</module>
\stopXML
}

\subsection[annotation-and-certainty]{Annotation and Certainty}

\subsection[exporting-data]{Exporting Data}

\stoptext
\stopcomponent