\startcomponent[AdvancedProgramming]
\starttext

\section[advanced-faims-programming]{Advanced FAIMS Programming}

\subsection[{\tt module.xml}-cheat-sheet]{{\tt module.xml} Cheat Sheet}

For more information about the different XML attributes, flags, and relationship tags, we have a README file that you can access online here: \useURL[url24][https://github.com/FAIMS/FAIMS-Tools/blob/master/generators/christian/readme][][{\em https://github.com/FAIMS/FAIMS-Tools/blob/master/generators/christian/readme}]\from[url24] or in the generators/christian/ directory where you downloaded the FAIMS-Tools.

We'll repeat some of the information here for your reference. Be sure to open the README for the most up to date information: to learn how to include more advanced controls and scripting in your modules, look at the \useURL[url25][https://faimsproject.atlassian.net/wiki/display/FAIMS/FAIMS+Data\%2C+UI+and+Logic+Cook-Book][][{\em FAIMS Development Cookbook}]\from[url25], which includes code snippets for all of the things FAIMS can do.

\copypages[readme/readme.pdf][n=11][scale=875]

\stoptext
\stopcomponent