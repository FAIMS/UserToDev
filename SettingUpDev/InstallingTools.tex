\startcomponent[InstallingTools]
\starttext

\section[findingtext]{Finding the right Text Editor for your}


The purpose of the FAIMS tools are to remove the need to individually craft all of your module's necessary files. Instead, you'll need to create one file the tools can use as a blueprint for making everything else. In order to make that one file (and to make any fancy edits to the necessary files to add detail or functionality), you're going to need a simple text editor that's good for coding with.

If you're doing your coding within the virtual machine, we recommend using an editor called gEdit which came with your Ubuntu installation. It's simple, functional, and won't clog up your instructions with unnecessary formatting like a standard Word Processor\footnote{
Note from your friends at FAIMS: One of the main reasons to use the text editors above is so that you can avoid \quotation{Smart Quotes,} an automatic feature of many word processor programs that will insert differently styled or extra quotes that will cause the FAIMS server to reject your module. If you run into any issues with your module, ensure that \quotation{Smart Quotes} are disabled in your text editor.}.

If you decide to write the files outside your virtual machine, we don't recommend using the text editors (such as Notepad) that come with Windows or OSX. While they may not complicate your code like a word processor would, they also don't have little features like autocomplete (a feature that means you don't need to fully type out words you use often) and syntax highlighting (a feature that helps you keep track of how your code is structured with helpful visual cues). With a little searching around, you can find plenty of excellent free or commercial text editors.

We particularly recommend:

\startitemize[packed,joinedup]
  \item \useURL[url11][https://notepad-plus-plus.org/download/v6.8.1.html][][{\em Notepad++}]\from[url11] (Windows)
  \item \useURL[url12][https://atom.io/][][{\em Atom}]\from[url12] (Cross Platform)
  \item \useURL[url13][http://www.barebones.com/products/textwrangler/][][{\em TextWrangler}]\from[url13] (OSX)
\stopitemize


For each of these, download their installers and follow their installation wizards. If you're using a Linux device that isn't the virtual machine, you can install gEdit, Vim, Emacs, or another text editor using apt-get.

% \section{Access to local files via VirtualBox}

%TODO This section seems important. And blank. Ruts, ideas?


\section[installing-the-android-sdk]{Installing the Android SDK}

Now we'll need to install the Android Software Development Kit (ADK). It's available at \useURL[url14][https://developer.android.com/sdk/installing/index.html][][\tt\tfx\hyphenatedurl{https://developer.android.com/sdk/installing/index.html}]\from[url14] Ignore the Android Studio link; we just need the SDK Tools.

Because you're going to use the tools for debugging an Android device plugged into your actual computer, you're going to want to download and install SDK Tools to your regular computer system, NOT your Ubuntu virtual machine.



\placefigure[force][click-standalone]{Click Stand-Alone SDK Tools.}{\externalfigure[image41.png][width=.5\textwidth]}    

\placefigure[force][download-sdk]{Choose to download the SDK Now.}{\externalfigure[image36.png][width=.5\textwidth]}

\placefigure[force][sdk-only]{Make sure to get the SDK tools only. We won't need to compile a new Android application here.}{\externalfigure[image60.png][width=.5\textwidth]}


Run the downloaded installer and follow its prompts to finish setting up the SDK tools.

\subsection[installing-the-faims-debug-apk-on-your-android-device]{Installing the FAIMS Debug APK on your Android Device}

The final step to setting up your development environment is to install the FAIMS app on your Android tablet or phone. Instead of installing the app through the Google Play Store, navigate to \useURL[url15][https://www.fedarch.org/apk/faims-debug-latest.apk][][{\tt\tfx\hyphenatedurl{https://www.fedarch.org/apk/faims-debug-latest.apk}}]\from[url15] and download the special debug version.

Next, you'll need to place the FAIMS APK file on your phone, either by moving it over via USB connection or by placing the APK file in a cloud service like Dropbox and downloading it through the mobile app.

If you load the URL \useURL[url16][https://www.fedarch.org/apk/faims-debug-latest.apk][][{\tt\tfx\hyphenatedurl{https://www.fedarch.org/apk/faims-debug-latest.apk}}]\from[url16] on your Android devices, you'll be able to directly download the file.

To install the APK, you may need to change the security settings on your device to allow non-market apps. Go to the \quotation{Settings} app, find the Security settings, and check \quotation{Unknown sources} (device pictured in \in{Figure}[security-change] is a Samsung Galaxy s5. Your settings menu may look different).

\placefigure[force][security-change]{Change your security settings to allow \quotation{Unknown sources}.}{\externalfigure[image62.png][width=.5\textwidth]}

\subsection[making-your-android-device-communicate-with-your-computer-with-usb-debugging]{Making your Android Device Communicate with your Computer with USB Debugging}

First, you'll need to enable the \quotation{Developer Options} on your device. This is another step where the instructions are different depending on what kind of device you're using. 

On Android versions 6+ to enable the \quotation{Developer Options}:

\startitemize[n]
\item
    Open the App drawer.
\item
    Launch Settings menu.
\item
    Find the open the \quotation{About Device} menu.
\item
    Scroll down to \quotation{Build Number}.
\item
    Next, tap on the \quotation{Build Number} section seven times. (Yes, really.)
\item
    After the seventh tap you will be told that you are now a developer. (Again, this is really how it works.)
\item
    Go back to Settings menu and the Developer Options menu will now be displayed.
\stopitemize 

You will need to install (for some samsung models) specific debug drivers:

Note that if you want for USB Debugging to work when your Samsung device is connected with your computer, then you will first need to make sure that the \useURL[url17][http://www.android.gs/download-samsung-usb-drivers-for-android/][][Samsung USB Drivers are installed]\from[url17] on your PC.


For many non-Samsung devices, you can use Google's ABD/USB driver instead: \useURL[url19][https://developer.android.com/studio/run/oem-usb.html][][\tt\tfx\hyphenatedurl{https://developer.android.com/studio/run/oem-usb.html}]\from[url19]

In order to enable the USB Debugging you will need to open Developer Options, scroll down and tick the box that says \quotation{USB Debugging}.

That's it! Your FAIMS development environment should now be set up properly.

\quiz{Test your knowledge of all of FAIMS needed support tools.}{
\item What are two things you can do if you don't find an option or setting where we tell you it will be?
\item If the \quotation{Error: Network Device Not Found} error appears when you start VirtualBox, how do you fix it?
\item True or False: Microsoft Word is a good program to code your modules in.
\item Should you install your Android developer kit within the virtual machine? Why or why not?

\item Which of the following are true?

\startitemize
\item
    VirtualBox allows you to run a different operating system on your computer
\item
    VirtualBox connects to a computer at the FAIMS offices
\item
    You must install XSLT Proc and saxon-xslt before running the XML generator tool
\item
    FAIMS can use both Android and Apple devices
\stopitemize

\item
    Exercise: Open up a command window. In this command window, write {\tt adb --version} then enter. Some information about the Android Development Bridge (ADB) program should appear. Did you encounter any errors? If not, congratulations, ADB/Android SDK is successfully installed.
}

\stoptext
\stopcomponent