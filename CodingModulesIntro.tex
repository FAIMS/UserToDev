\startcomponent[codingmodulesintro]
\starttext

For this tutorial, we'll work together and create a simple FAIMS module. For illustrative purposes, we're going to recreate a simple module available from the FAIMS demo server--the "Oral History" module.

We're going to start by showing you how to make a basic but fully functional version of the module that can be made only by constructing a relevant module.xml. Then, after you've run that through the FAIMS-tools and produced a fully functioning but simple version, we'll teach you some ways to add complexity and functionality by modifying the necessary files directly.

\section{Introduction to Module.xml}

Earlier sections introduced the basic necessary files of a module. You should already have some idea what these were:

\startitemize[n]
\item a Data Definition Schema (data_schema.xml).
\item a User Interface Schema (ui_schema.xml)
\item a User Interface Logic file (ui_logic.bsh)
\item a Translation (Arch16n) file (faims.properties)
\item A server-side validation file (validation.xml)
\item CSS Files for styling the modules (ui_??)
\stopitemize

You've already learned that you don't have to design and code all of these files from scratch. Instead, you'll create one file that serves as a set of instructions for the FAIMS Tools to create the necessary files for you. That one file is called module.xml.

\quiz{Test your knowledge: a gentle descent into module.xml}{
  \item What are some programs you can use when modifying module.xml? Which programs should you avoid? If you don't remember, review the previous section, "Setting Up Your Development Environment."
}

\stoptext
\stopcomponent