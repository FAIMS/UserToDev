\setupcolors[state=start]


% \definetextbackground[secondary][
%         width=.8\makeupwidth,
%         background=color,backgroundcolor=gray,
%         topoffset=.5\bodyfontsize,bottomoffset=.5\bodyfontsize,
%         frame=off]

%%%%%%%%%%%%%% with FLOAT %%%%%%%%%%%%%%%

% \define[3]\foo{
% \placecode[here][#1]{#2}{
% 	\startsecondary
% 	{#3}
% 	\stopsecondary
% }
% }

%%%%%%%%%%%%% code block testing %%%%%%%%%%%%%

\definefloat[code]

\setupcaption[code][location=below, width=.8\textwidth]
%TODO Figure out how to make smaller and indented.

\define[3]\wibblecode{%
\placecode[here][#1]{#2}{
\startframedtext[align={hanging, hz},
				width=\makeupwidth,
				background=color,
				backgroundcolor=gray,				
				framecolor=gray,
				rulethickness=1pt				
				 ]
				 
\starttyping[option=xml, tab=4, style={\tfx\tt}] 
#3
\stoptyping

\stopframedtext
}}


\starttext

\wibblecode{codeTitle}{Breaking down {\tt <Title>}.}{
<Title f="id notnull">
	<desc>This title should be a sensible title, unique to each item, briefly summarising the contents of the item, for example "Ilocano songs recorded in Burgos, Ilocos Sur, Philippines, 17 April 1993"</desc>
</Title>		
}

\stoptext





